% DO NOT EDIT - automatically generated from metadata.yaml

\def \codeURL{https://github.com/bruAristimunha/ReScience-submission}
\def \codeDOI{}
\def \dataURL{}
\def \dataDOI{}
\def \editorNAME{}
\def \editorORCID{}
\def \reviewerINAME{}
\def \reviewerIORCID{}
\def \reviewerIINAME{}
\def \reviewerIIORCID{}
\def \dateRECEIVED{}
\def \dateACCEPTED{}
\def \datePUBLISHED{}
\def \articleTITLE{[Re] Deep Convolution Neural Network and Autoencoders-Based Unsupervised Feature Learning of EEG Signals}
\def \articleTYPE{Replication}
\def \articleDOMAIN{}
\def \articleBIBLIOGRAPHY{bibliography.bib}
\def \articleYEAR{2020}
\def \reviewURL{}
\def \articleABSTRACT{This paper presents our efforts to implement, acquire similar results and improve the ones achieved by the authors of the original article. We follow the steps and models described in their article and the same public data sets of EEG Signals. Epilepsy affects more than 65 million people globally, and EEG Signals are critical to analyze and recognize epilepsy. Although the efforts in the last years, it is still challenging to extract useful information from these signals and select useful features from numerous them in a diagnostic application. We construct a deep convolution network and autoencoders-based model (AE- CDNN) in order to perform unsupervised feature learning. We use the AE- CDNN to extract the features of the available data sets, and then we use some common classifiers to classify the features. The results obtained demonstrate that the proposed AE-CDNN to perform the traditional feature extraction based classification techniques by achieving better accuracy of classification.}
\def \replicationCITE{}
\def \replicationBIB{}
\def \replicationURL{}
\def \replicationDOI{}
\def \contactNAME{Bruno Aristimunha}
\def \contactEMAIL{b.aristimunha@gmail.com}
\def \articleKEYWORDS{epilepsy, auto-enconder, EEG, python.}
\def \journalNAME{ReScience C}
\def \journalVOLUME{4}
\def \journalISSUE{1}
\def \articleNUMBER{}
\def \articleDOI{}
\def \authorsFULL{Bruno Aristimunha et al.}
\def \authorsABBRV{B. Aristimunha et al.}
\def \authorsSHORT{Aristimunha et al.}
\title{\articleTITLE}
\date{}
\author[1]{Bruno Aristimunha}
\author[1,]{Diogo Eduardo Lima Alves}
\author[1, 2,,\orcid{0000-0003-3739-1087}]{Walter Hugo Lopez Pinaya}
\author[1,,\orcid{0000-0001-6021-747X}]{Raphael Y. de Camargo}
\affil[1]{Center for Mathematics, Computation and Cognition (CMCC), Federal Univesity of ABC (UFABC), Rua Arcturus, 03. Jardim Antares, São Bernardo do Campo, CEP 09606-070, SP, Brazil.}
\affil[2]{Department of Psychosis Studies, Institute of Psychiatry, Psychology \& Neuroscience, King’s College London, London, UK}
